\documentclass{beamer}
\usepackage[english]{babel}
\usepackage{amsmath}
\usepackage{graphicx}
\usepackage{lmodern}
\usepackage[font=small,format=plain,labelfont=bf,up,textfont=it,up]{caption}
\usepackage{url}
\usepackage{courier}
\usepackage[T1]{fontenc}
\usepackage{subcaption}
\usepackage{listings}
\usepackage{xcolor}
\usepackage[plain,vlined]{algorithm2e}

\setcounter{secnumdepth}{5}
\setlength{\parindent}{0in}

\usetheme{Antibes}

\newcommand{\squeezeup}{\vspace{0mm}}

\renewcommand{\topfraction}{0.85}
\renewcommand{\textfraction}{0.1}
\renewcommand{\floatpagefraction}{0.75}

\definecolor{keyword}{RGB}{70,100,200}
\definecolor{kernel}{RGB}{139,0,0}
\definecolor{arrow}{RGB}{200,0,0}

\def\nese{\mathrel{%
    \mathchoice{\NESE}{\NESE}{\scriptsize\NESE}{\tiny\NESE}%
}}
\def\NESE{{%
    \setbox0\hbox{$\nearrow$}%
    \rlap{\hbox to \wd0{\hss $\searrow$ \hss}}\box0
}}

\def\nesecol{\mathrel{%
    \mathchoice{\NESECOL}{\NESECOL}{\scriptsize\NESECOL}{\tiny\NESECOL}%
}}
\def\NESECOL{{%
    \setbox0\hbox{$\nearrow$}%
    \rlap{\hbox to \wd0{\hss \textcolor{arrow}{$\searrow$} \hss}}\box0
}}

\def\nwsw{\mathrel{%
    \mathchoice{\NWSW}{\NWSW}{\scriptsize\NWSW}{\tiny\NWSW}%
}}
\def\NWSW{{%
    \setbox0\hbox{\textcolor{arrow}{$\nwarrow$}}%
    \rlap{\hbox to \wd0{\hss $\swarrow$ \hss}}\box0
}}

\title{Gene-Environment Interaction Analysis using GPU}
\author{Daniel Berglund}
\date{December 2014}

\begin{document}

\begin{frame}
 \titlepage
\end{frame}

\section*{Outline}
\begin{frame}
\frametitle{Outline}
 \tableofcontents
\end{frame}

\section{Background}

\begin{frame}
\frametitle{Problem and Aim}

\begin{itemize}
 \item Genetic and environment factors are know to affect the risks of diseases
 \item Interaction can exist between these factors
 \item Logistic Regression can be used to search for interaction
\begin{itemize}
 \item Iterative method
 \item Models hidden probabilities of the outcomes as a linear model
\end{itemize}
\end{itemize}

\end{frame}

\begin{frame}
\frametitle{Problem and Aim}

\begin{itemize}
 \item Data amounts are increasing
 \item Need for more speed
 \item GPUs have previously shown good results for gene-gene interaction
\end{itemize}

\end{frame}

%\begin{frame}
%\frametitle{The Data}
 
%\begin{itemize}
% \item The data consists of individuals
%\end{itemize}

%\end{frame}

%\begin{frame}
%\frametitle{Risk Allele}
 
%\begin{itemize}
% \item Asdf
%\end{itemize}

%\end{frame}

\begin{frame}
\frametitle{Recoding}

\begin{itemize}
 \item The measures for additive interaction are defined for positive odds ratios
 \item Can be adjusted by recoding
 \item Recoding switches the reference group with the group with lowest risks
 \item Guarantees that $OR \geq 1$
\end{itemize}

\end{frame}

%\subsection{Computer Architecture}

%lots of optimizations CPU picture
%knowing is imporant, take the matrix matrix multiplaction example

\begin{frame}
\frametitle{Concurrency and Dinning Philosophers}

\begin{figure}[h]
    \centering
    \includegraphics[width=5cm]{../dining_philosophers.png}
\end{figure}

\end{frame}

\begin{frame}
\frametitle{A Possible Solution} 

\begin{itemize}
 \item Think
 \item Wait for a fork to become available and pick up that fork
 \item Wait for and pick up the other fork
 \item Eat
 \item Put down the forks one by one
 \item Go back to thinking
\end{itemize}

\end{frame}

\begin{frame}
\frametitle{Deadlock}

\begin{itemize}
 \item Stuck when everyone holds one fork
 \item This is a deadlock
 \item Can be solved with a mediator
\end{itemize}

\end{frame}

\begin{frame}
\frametitle{Concurrency Architectures}
 \begin{figure}
        \centering
        \begin{subfigure}[b]{0.48\textwidth}
                \includegraphics[width=3.5cm]{../SISD}
                \caption*{SISD}
        \end{subfigure}%
        ~
        \begin{subfigure}[b]{0.48\textwidth}
                \includegraphics[width=3.5cm]{../MISD}
                \caption*{MISD}
        \end{subfigure}

        \begin{subfigure}[b]{0.48\textwidth}
                \includegraphics[width=3.5cm]{../SIMD}
                \caption*{SIMD}
        \end{subfigure}
        ~
        \begin{subfigure}[b]{0.48\textwidth}
                \includegraphics[width=3.5cm]{../MIMD}
                \caption*{MIMD}
        \end{subfigure}
\end{figure}
\end{frame}

\begin{frame}
\frametitle{GPU Architecture}

\begin{figure}[h]
    \centering
    \includegraphics[width=3.6cm]{../gpu_scheme.png}
    \label{fig:gpu_scheme}
\end{figure}

\end{frame}

\begin{frame}
\frametitle{GPU Architecture}

\begin{itemize}
 \item Separate memory from the normal RAM
 \item Data has to be explicitly transferred to and from the GPU
 \item Simpler cores than CPU cores
 \item Single instruction, multiple thread(SIMT)
\end{itemize}

\end{frame}

\begin{frame}
\frametitle{CUDA}

\begin{itemize}
 \item For NVIDIA GPUs by NVIDIA
 \item Supported by various libraries, CUBLAS, ArrayFire, and so on
 \item Calculations done by kernels
\end{itemize}

\end{frame}

\begin{frame}
\frametitle{Kernel Example}

\begin{algorithm}[H]
\DontPrintSemicolon
\SetStartEndCondition{ (}{)}{)}\SetAlgoBlockMarkers{\{}{\}}
\SetKwProg{Fn}{}{}{}\SetKwFunction{FRecurs}{void FnRecursive}
\SetKwFor{For}{for}{}{}
\SetKwIF{If}{ElseIf}{Else}{if}{}{elif}{else}{}
\SetKwFor{While}{while}{}{}
\SetKwRepeat{Repeat}{repeat}{until}
\AlgoDisplayBlockMarkers
\SetAlgoNoLine
\SetFuncSty{}
\SetArgSty{}

\BlankLine \BlankLine

\SetKwFunction{KwFn}{\textcolor{kernel}{\_\_global\_\_} void \KwSty{Add}}

\Fn(){\KwFn{\textcolor{keyword}{float}* $A$, \textcolor{keyword}{float}* $B$, \textcolor{keyword}{float}* $C$}}{
\textcolor{keyword}{int} $i$ = threadIdx.$x$;\;
$C[i]$ = $A[i]$ $+$ $B[i]$;\;
}

\end{algorithm}

\begin{algorithm}[H]
\DontPrintSemicolon
\SetStartEndCondition{ (}{)}{)}\SetAlgoBlockMarkers{\{}{\}}%
\SetKwProg{Fn}{}{}{}\SetKwFunction{FRecurs}{void FnRecursive}%
\SetKwFor{For}{for}{}{}%
\SetKwIF{If}{ElseIf}{Else}{if}{}{elif}{else}{}%
\SetKwFor{While}{while}{}{}%
\SetKwRepeat{Repeat}{repeat}{until}%
\AlgoDisplayBlockMarkers\SetAlgoNoLine%

\BlankLine \BlankLine

\textbf{Add}\textcolor{kernel}{$<<<N$,$M>>>$}($A, B, C$);\;

\end{algorithm}

\end{frame}

\begin{frame}
\frametitle{Blocks}

\begin{figure}[h]
    \centering
    \includegraphics[width=7cm]{../2D_grid.png}
\end{figure}

\end{frame}

\begin{frame}
\frametitle{Blocks}

\begin{figure}[h]
    \centering
    \includegraphics[width=7cm]{../grid_scale.png}
\end{figure}

\end{frame}

\begin{frame}
\frametitle{Streams}

\begin{itemize}
 \item Kernels and transfers can be performed on streams
 \item Transfers and kernels can overlap when using streams
\end{itemize}

\end{frame}

\begin{frame}
\frametitle{Streams}

\begin{figure}[h]
    \centering
    \includegraphics[width=7cm]{../StreamsQueue.png}
\end{figure}

\end{frame}

\begin{frame}
\frametitle{Streams}

\begin{figure}
        \centering
        \begin{subfigure}[b]{0.48\textwidth}
        \centering
                \includegraphics[width=0.9cm]{../StreamsQueueStreamLoop.png}
                \caption*{Looped over stream}
        \end{subfigure}%
        ~
        \begin{subfigure}[b]{0.48\textwidth}
        \centering
                \includegraphics[width=0.9cm]{../StreamsQueueKernelLoop.png}
                \caption*{Looped over kernel}
        \end{subfigure}
\end{figure}

\end{frame}

\section{Implementation}

%\begin{frame}
%\tableofcontents[currentsection]
%\end{frame}

%\subsection{Clean Code and Software Design}

%\begin{frame}
%\frametitle{Repeatability in Computer Science}

%\begin{figure}[h]
%    \centering
%    \includegraphics[width=9cm]{../comp_repro.png}
%    \label{fig:speed}
%\end{figure}

%\end{frame}

\begin{frame}
\frametitle{Agile Development and Clean Code}
\begin{itemize}
 \item Code in small units, commonly classes
 \item Unit testing, test the units in isolation
 \item Mocks, i.e. fake objects
 \item If a unit is working incorrectly only its corresponding unit test should fail
 \item Integration tests can be used to test if the units work properly together
\end{itemize}

\end{frame}

%why clean code/good code/testing
%because of the previsous and maintainability
%need to trust the code, check math, data quality etc, but not code quality
%not all errors causes fatal or obvious wrong results
%example of the matrix thing I did, y=ax*x+y

%Wrappers
\begin{frame}
\frametitle{Wrappers}

\begin{itemize}
 \item Wrappers are used to ``fix'' interfaces
 \item Does not perform the task, delegates it
\end{itemize}

\end{frame}

\begin{frame}
\frametitle{Wrappers}

\begin{algorithm}[H]
\DontPrintSemicolon
\SetStartEndCondition{ (}{)}{)}\SetAlgoBlockMarkers{\{}{\}}%
\SetKwProg{Fn}{}{}{}\SetKwFunction{FRecurs}{void FnRecursive}%
\SetKwFor{For}{for}{}{}%
\SetKwIF{If}{ElseIf}{Else}{if}{}{elif}{else}{}%
\SetKwFor{While}{while}{}{}%
\SetKwRepeat{Repeat}{repeat}{until}%
\AlgoDisplayBlockMarkers\SetAlgoNoLine%

\textbf{cublasSgemv}($cublasHandle$, $transpose$, $m$, $n$, $\alpha$, $matrix$, $ld\_matrix$, $vector$, $inc\_vector$,
      $\beta$, $result$, $inc\_result$);

\end{algorithm}

\begin{algorithm}[H]
\DontPrintSemicolon
\SetStartEndCondition{ (}{)}{)}\SetAlgoBlockMarkers{\{}{\}}%
\SetKwProg{Fn}{}{}{}\SetKwFunction{FRecurs}{void FnRecursive}%
\SetKwFor{For}{for}{}{}%
\SetKwIF{If}{ElseIf}{Else}{if}{}{elif}{else}{}%
\SetKwFor{While}{while}{}{}%
\SetKwRepeat{Repeat}{repeat}{until}%
\AlgoDisplayBlockMarkers\SetAlgoNoLine%

\BlankLine \BlankLine

\textbf{matrixVectorMultiply}($matrix$, $vector$, $result$, $\alpha$, $\beta$);

\end{algorithm}

\end{frame}

\begin{frame}
\frametitle{Structure of Program}

\begin{itemize}
 \item Modularised
 \item Almost full unit test coverage
 \item Handles each gene environment combination independently
 \item Threads fetches a genetic factor from a queue
 \item Each thread corresponds to a stream on a GPU
 \item Some parts are done using CPU

\end{itemize}

\end{frame}

\begin{frame}
\frametitle{Model}

\begin{figure}[h]
    \centering
    \includegraphics[width=7cm]{../lr_structure.png}
\end{figure}

\end{frame}

\section{Results}

%\begin{frame}
%\tableofcontents[currentsection]
%\end{frame}

%\subsection{Streams}

\begin{frame}
\frametitle{Streams}

\begin{figure}[h]
    \centering
    \includegraphics[width=7cm]{../gpu_10streams_10ks_100ki.png}
\end{figure}

\end{frame}

%\subsection{Scaling over Multiple GPUs}

\begin{frame}
\frametitle{Saturated Streams}

\begin{figure}[h]
    \centering
    \includegraphics[width=7cm]{../saturated_seconds_ind_cov_0.png}
\end{figure}

\end{frame}

\begin{frame}
\frametitle{Speedup and Efficiency}

\begin{equation*}
S(p)=\frac{T(1)}{T(p)}
\end{equation*}

\begin{equation*}
E(p)=\frac{S(p)}{p}=\frac{T(1)}{pT(p)}
\end{equation*}

\end{frame}

\begin{frame}
\frametitle{Speedup and Efficiency}

Versus one GPU

\begin{figure}
        \centering
        \begin{subfigure}[b]{0.48\textwidth}
                \includegraphics[width=5cm]{../saturated_speedup_individuals_cov_0.png}
        \end{subfigure}%
        ~
        \begin{subfigure}[b]{0.48\textwidth}
                \includegraphics[width=5cm]{../saturated_efficiency_individuals_cov_0.png}
        \end{subfigure}
\end{figure}

\end{frame}

\begin{frame}
\frametitle{Time Distribution for Kernels}

{\tiny $*$ is element by element multiplication}

\squeezeup

\begin{table}[h]
\centering
\begin{tabular}{| l | l | c | c | c | c |}
  \hline
  Covariates & & 0 & 0 & 20 & 20 \\
  Individuals & & 10k & 100k & 10k  & 100k  \\
  \hline
  $M1^T \cdot V1$ & CUBLAS & 66.6 & 72.2 & 43 & 53.9\\
  $M1 \cdot M2$ & CUBLAS & 18.9 & 14.4 & 36.1 & 25.8\\
  $M1 \cdot V1$ & CUBLAS & 7.1 & 8.1 & 5.9 & 7.4\\
  $V1 * V2$ & Custom & 2.9 & 2.3 & 11.8 & 10.6\\
  $\frac{e^{V1}}{1+e^{V1}}$ & Custom & 1.1 & 0.9 & 0.7 & 0.7\\
  %$V1*\log(V2)+$ \newline $(1-V1)*\log(1-V2)$
  
{$\begin{aligned}
& V1*\log(V2)+\\
& (1-V1)*\log(1-V2)
\end{aligned}$}
  & Custom & 0.8 & 0.6 & 0.6 & 0.5\\
    
  $V1 - V2$ & Custom & 0.7 & 0.6 & 0.5 & 0.4\\
  $V1*(1-V1)$ & Custom & 0.7 & 0.5 & 0.4 & 0.4\\
  \emph{Dot product} & CUBLAS & 0.7 & 0.3 & 0.3 & 0.2\\
  $\Sigma V1$ & CUBLAS & 0.5 & 0.1 & 0.2 & 0.1\\
  \hline  
\end{tabular}
\end{table}

\end{frame}

%\subsection{Single Versus Double Precision}

\begin{frame}
\frametitle{Single Versus Double Precision}

\begin{figure}
        \centering
        \begin{subfigure}[b]{0.48\textwidth}
        \centering
                \includegraphics[width=5cm]{../double_comp_cov_ind_10000.png}
                \caption*{10000 individuals}
        \end{subfigure}%
        ~
        \begin{subfigure}[b]{0.48\textwidth}
        \centering
                \includegraphics[width=5cm]{../double_comp_ind_cov_0.png}
                \caption*{0 covariates}
        \end{subfigure}
\end{figure}

\end{frame}

%\subsection{Syncing}

\begin{frame}
\frametitle{Syncing}

\begin{figure}
        \centering
        \begin{subfigure}[b]{0.48\textwidth}
                \includegraphics[width=5cm]{../sync_comp_cov_ind_2000.png}
                \caption*{2000 individuals}
        \end{subfigure}%
        ~
        \begin{subfigure}[b]{0.48\textwidth}
                \includegraphics[width=5cm]{../sync_comp_cov_ind_2e+05.png}
                \caption*{200 000 individuals}
        \end{subfigure}
\end{figure}

\end{frame}

\begin{frame}
\frametitle{Synchronisation}

\begin{figure}
        \centering
        \begin{subfigure}[b]{0.48\textwidth}
                \includegraphics[width=5cm]{../timedist_LR_ind_10ks_0cov_gpu4.png}
                \caption*{Less synchronisation}
        \end{subfigure}%
        ~
        \begin{subfigure}[b]{0.48\textwidth}
                \includegraphics[width=5cm]{../timedist_LR_ind_10ks_0cov_gpu4_sync.png}
                \caption*{Always synchronise}
        \end{subfigure}
\end{figure}

\end{frame}

\section{Conclusions and Outlook}

\begin{frame}
\frametitle{Conclusions}

\begin{itemize}
 \item GPUs suits well for interaction, depending on method
 \item Use one, perhaps two, GPUs with the program
 \item Single precision gives better performance than double precision
 \item Increased synchronisation increases performance slightly when number of individuals is above 100 000
\end{itemize}

\end{frame}

\begin{frame}
\frametitle{Outlook}

\begin{itemize}
 \item Moving more parts to the GPU could fix the bad scaling
 \item Clusters for more speed
 \item Need for better statistics and definitions of interaction for non binary factors
 \item For binary factors look into gene-gene interaction methods if further speed is needed
\end{itemize}

\end{frame}

%TODO questions slide

\begin{frame}
\frametitle{Matrix Decomposition}

\begin{itemize}
 \item Pseudo inverse, $A^+$, is defined for general matrices
 \item Can be found by using singular value decomposition
\end{itemize}

\begin{equation*}
A=U \Sigma V^T
\end{equation*}

\begin{equation*}
A^+=V \Sigma^+ U^T
\end{equation*}

\end{frame}

\begin{frame}
\frametitle{Odds, Odds Ratio and Additive Interaction}

\begin{equation*}\label{eq:odds}
\Omega=\frac{\pi}{1-\pi}
\end{equation*}

\begin{equation*}\label{eq:odds_ratio}
\theta=\frac{\Omega_1}{\Omega_2}
\end{equation*}

\begin{equation*}
\theta_{both\:factors\:present}>\theta_{first\:factor\:present}+\theta_{second\:factor\:present}-1
\end{equation*}

\end{frame}

%TODO make slides about possible questions

%opencl vs cuda

\end{document}